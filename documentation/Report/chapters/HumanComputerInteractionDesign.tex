\chapter{Human-Computer Interaction Design}
The application is highly complex because of the nature of the application and its many functionalities.To assert the quality of interaction with the user and to detect flaws in the interaction design, an experiment has been conducted that tests the users’ ability to find and use the features of the application to correctly and easily derive conclusions.

\section{Method}
In this section a description can be found on what the experiment consists of and how the experiment should be performed, in such a way that the result should be reproducible.

\section{Target}
The goal of the experiment is to assert the test users’ ability to intuitively understand the graphical user interface and, in particular, understand the representation of the research data that is used in the application.

\section{Procedure}
The experiment will consist of two stages. In the first stage, the participants are shortly trained in the basic biology principles necessary to understand the tasks they are given, as well as explained what the data is and what it represents.
In the second stage, the participants spend 15 minutes performing a small list of tasks inside the application, which mimic the expected research questions that the end-user may have. This stage is executed individually with the researcher.
In the third stage, the test users are interviewed to reflect on their experience and the problems they encountered while using the application.

\par
The following tasks as set on the user. Each task is provided after the previous has been finished.

\begin{enumerate}
	\item Produce the sequence graph of the complete data set.
	\item Find the position of the gene ‘katG’ in the DNA.
	\item Decide if there are any mutations in the gene.
	\item If any mutations are present, find out what nucleotides differ between the mutations.
\end{enumerate}

\section{Tasks}
This usability test should be performed by at least 2 persons: the researcher and the participant. Because the whole test will be captured there is no need for more than one person to analyse the user, as the researcher can have another look at the experiment afterwards.
Measurement
\par
The duration of each task is measured and the researcher takes notes of interactions that take significantly longer than expected. If the participant takes longer than the maximum amount of time to complete one of the tasks, the researcher provides a hint to steer the participant towards the answer. This signifies that the participant was not able to find the functionality required to complete the task autonomously. This can point to certain aspects of that task being unclear, unavailable, or insufficiently useful.
In addition, the tasks that require the participant to derive a conclusion, are marked correct or incorrect.

\section{Participants}
The test users are currently students in biology related areas, such as medicine or (micro)biology. These backgrounds provide general understanding of DNA and proteins, but not necessarily about the bioinformatical aspects of DNA sequencing and sequence alignment. This knowledge is provided during the training.
\section{Result}
We provide the results of the measurements for each step in appendix \ref{ch:human-computer-interaction-design-results}. 
